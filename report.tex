\documentclass{article}
\usepackage{csquotes}

\title{ELEC6242 Coursework Cryptoanalysis}
\author{Emily Shepherd \\ ams2g11@ecs.soton.ac.uk}
\date{18th April 2016}
\begin{document}
    \maketitle

    \section{Overview}

    This report sets provides the solutions to the two indidivual
    ciphers set for the ELEC6242 Coursework challenge, as well as
    detailing the process which was used to crack these.

    \section{Cipher 1}

    \subsection{Solution}

    \begin{displayquote}
        The findings could lead to a better understanding of black hole
        mergers and galaxy evolution, and also help shed light on a
        long-standing conundrum in astrophysics called the "final
        parsec problem." That refers to the failure of theoretical
        models to predict what the final stages of a black hole merger
        look like, or even how long the process might take.
    \end{displayquote}

    \subsection{Approach}

    When first inspecting this cipher, my initial observation was that
    it appeared as though spaces and other punctuation had been
    preserved. This was because full stops and commas appear only
    directly after a word, and sentences are all multiple words long.
    The first letter of each sentence was also capitalised, and
    quotation marks appeared to be logical. From this, I deduced that
    the first course of investigation should assume that letters had
    also not been moved, so I focused on substitution ciphers, rather
    than transposition ciphers.

    My first tactic was to use frequency analysis on the cipher; this
    gave a result which was close to what would be expected from
    English text. Although it was unclear if this was acceptable error,
    or if there was more than one key in use, I decided to use perform
    this analysis with various key lengths to check. Higher key lengths
    of around 2-4 gave stronger results, leading me to suspect further
    that a multiple shifts were being used.

    Due to this conclusion, I suspected the Vigenere chipher may be in
    use. As such, the next stage was to perform the Kasisky test to
    attempt to find the key length. Using a scipt, I searched for
    repeated words, and matched their distances between them. The main
    words of interest were: xya and wru, both of which appeared three
    times, each time at a distance a multiple of three.

    From inspection, there are two one letter words in the cipher text:
    e and r. The distance between the "e"s is a multiple three, whoever
    this is not the case for the "r", suggesting the "r" corresponds to
    a different letter in the key. As there are only two single letter
    words in the English Language ("I" and "a"), the shift values could
    be obtained from trail and improvement - I started by assuming both
    of these were "a", as they were both lowercase. This turned out to
    be correct, and revealed two thirds of the message. From there,
    guessing the final shift was simple: as the text started with "?he",
    I attempted a shift to create "The". This gave the fully decrypted
    text.

\end{document}
